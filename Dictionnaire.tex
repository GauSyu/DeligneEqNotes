% !Mode:: "TeX:UTF-8"
% @ Commone Head File
\input{ArticleHead}
% @ Theorem Environments
\input{Theorem/margin-section}
\input{Head-after}
\input{Theorem/theoremlists}
%
% PDF File Information
%
\hypersetup{
pdftitle={Deligne: Equations Différentielles à Points Singuliers Réguliers},%标题
pdfauthor={Xu Gao},            %作者
%             pdfproducer={XeLaTeX},        %制作工具
bookmarksopen=true,         %书签自动打开
colorlinks=true,                       %是否采用彩色超链接
citecolor=red,                       %文献引用的颜色
filecolor=black,                       %文件链接颜色
linkcolor=black,                       %内部链接颜色
urlcolor=darkgray                         %网页与电邮链接颜色
}

%
% TITLE
%
\title{
Deligne's  \bigskip \\
\texttt{\Huge Equations Diff\'{e}rentielles \`{a} Points Singuliers R\'{e}guliers}\\
Part I: Dictionnaire
}
\author{
Note: Xu Gao
}
\date{
Last update:\today
}

\begin{document}
\maketitle
\begin{abstract}
  This is a reading note of Part I (\emph{Dictionnaire}) of Deligne's
  ``\emph{Equations Diff\'{e}rentielles \`{a} Points Singuliers R\'{e}guliers}''.
  Although closely follows the original French text, it is not a faithful
  English-translation: some supplementary materials are inserted and the
  numbering is thus different.
\end{abstract}

\tableofcontents
\clearpage

\section{Local systems and fundamenal group}

\begin{definition}
  Let $X$ be a topological space. A \itblue{(complex) local system} on $X$ is a
  sheaf of complex vector spaces on $X$ which, locally on $X$, is isomorphic to
  the constant sheaf $\underline{\CC^n}$ ($n\in\NN$).
\end{definition}

\begin{para}\label{hypothesis:1.2}
  Let $X$ be a \emph{locally path-connected} and
  \emph{locally simply connected}
  topological space, equipped with a base point $x_0\in X$.

  Let $\Ff$ be a locally constant sheaf on $X$.
  For each path $\alpha\colon[0,1]\to X$, the pullback $\alpha^{\ast}\Ff$
  of $\Ff$ on $[0,1]$ is a locally constant sheaf, hence constant and there
  exists a quniue isomorphism between $\alpha^{\ast}\Ff$ and the constant
  sheaf defined by the set $(\alpha^{\ast}\Ff)_0 = \Ff_{\alpha(0)}$.
  This isomorphism defines an isomorphism $\alpha(\Ff)$ between
  $(\alpha^{\ast}\Ff)_0$ and $(\alpha^{\ast}\Ff)_1$, i.e. an isomorphism
  \[
    \alpha(\Ff)\colon \Ff_{\alpha(0)}\To\Ff_{\alpha(1)}.
  \]
  This isomorphism depends only on the homotopy class of $\alpha$ and satisfies
  $\alpha\beta(\Ff)=\alpha(\Ff)\circ\beta(\Ff)$.
  In particular, $\pi_1(X,x_0)$ acts (on the left) on the stalk $\Ff_{x_0}$ of
  $\Ff$ at $x_0$.
\end{para}

\begin{proposition}\label{prop:equivalence_monodromy}
  Under the hypothesis \ref{hypothesis:1.2}, with $X$ being connected,
  the functor $\Ff\mapsto\Ff_{x_0}$ is an equivalence between
  the category of locally constant sheaves on $X$ and
  the category of left $\pi_1(X,x_0)$-sets.
\end{proposition}

\begin{subpara}\label{def:monodromy_action}
  Let $p\colon Y\to X$ be a covering space.
  For each path $\alpha\colon[0,1]\to X$ and
  $y\in Y_{\alpha(0)}:=p^{-1}(\alpha(0))$,
  there exists a unique path $\widetilde{\alpha}\colon[0,1]\to Y$ such that
  $p\circ\widetilde{\alpha}=\alpha$ and $\widetilde{\alpha}(0)=y$.
  Thus $\alpha.y:=\widetilde{\alpha}(1)\in Y_{\alpha(1)}$.
  This construction defines a bijective map from $Y_{\alpha(0)}$
  to $Y_{\alpha(1)}$ and depends only on the homotopy class of $\alpha$
  and satisfies $\alpha\beta.y=\alpha.(\beta.y)$.
  In particular, it defines an left action (the \itblue{monodromy action})
  of $\pi_1(X,x_0)$ on the fiber $Y_{x_0}$.
\end{subpara}

\begin{Remark}
  Let $p\colon Y\to X$ be a covering space and $\alpha$ be a path in $X$.
  Then the pullback $\alpha^{\ast}Y$ of $Y$ along $\alpha$ is a covering space
  of $[0,1]$, hence trivial.
  \[
  \begin{tikzcd}
    \alpha^{\ast}Y\ar[r,"{\alpha'}"]\ar[d,"{p'}"] & Y\ar[d,"p"]\\
    {[0,1]}\ar[r]\ar[u,bend left,dashed] & X
  \end{tikzcd}
  \]
  For each point $y$ in the fiber $Y_{\alpha(0)}$,
  there is a unique preimage $y'$ in the fiber $(\alpha^{\ast}Y)_0$.
  As $p'$ is a trivial covering, such a point uniquely determines a section of
  $p'$. It is clear that the composition of this section with $\alpha'$ gives
  a lifting of $\alpha$ into $Y$ starting from $y$ and any such lifting is given
  in this way.
\end{Remark}

\begin{subproposition}
  Under the hypothesis \ref{hypothesis:1.2}, with $X$ being connected,
  the monodromy actions define an equivalence bewteen
  the category of covering spaces of $X$ and
  the category of left $\pi_1(X,x_0)$-sets.
\end{subproposition}

\begin{subpara}
  Let $\Ff$ be a sheaf on $X$. We construct a space $X_{\Ff}$ over $X$
  as follows:
  the underlying set is the disjoint union of all stalks of $\Ff$ and
  the projection is induced by the maps $\Ff_x\to\{x\}$;
  the topology is the coarsest one in which every section $s\in\Ff(U)$ gives
  an open set $s(U):=\{(s_x,x):x\in U\}$.
  When $\Ff$ is locally constant, this construction gives a covering space of
  $X$ whose sheaf of sections is isomorphic to $\Ff$.
\end{subpara}

\begin{subproposition}
  The functor $\Ff\to X_{\Ff}$ is an equivalence between
  the category of locally constant sheaves and
  the category of covering spaces of $X$.
\end{subproposition}

Combining the above two propositions, after verfying the composition of those
equivalences cocides with the functor $\Ff\mapsto\Ff_{x_0}$, the statement
follows.

\begin{corollary}\label{cor:1.4}
  Under the hypothesis \ref{hypothesis:1.2}, with $X$ being connected,
  the functor $\Ff\mapsto\Ff_{x_0}$ is an equivalence between
  the category of local systems on $X$ and
  the category of finite-dimensional complex representations of $\pi_1(X,x_0)$.
\end{corollary}

\begin{para}\label{hypothesis:1.5}
  Under the hypothesis \ref{hypothesis:1.2}, if $\alpha$ is a path and $\beta$
  a loop from $\alpha(0)$, so $\alpha(\beta):=\alpha\beta\alpha^{-1}$ is a loop
  from $\alpha(1)$ and its homotopy class depends only on those of $\alpha$ and
  $\beta$. This construction defines an isomorphism between $\pi_1(X,\alpha(0))$
  and $\pi_1(X,\alpha(1))$.
\end{para}

\begin{proposition}\label{prop:fundamental_groupoid}
  Under the hypothesis \ref{hypothesis:1.5}, there exists,
  uniquely up to a unique isomorphism, a locally constant sheaf of groups
  $\Pi_1(X)$ on $X$ (the \itblue{fundamental groupoid}), equipped with,
  for every $x_0\in X$, an isomorphism
  \begin{equation}\label{eq:1.6.1}
    {\Pi_1(X)}_{x_0}\Isom\pi_1(X,x_0)
  \end{equation}
  such that, for any path $\alpha$,
  the isomorphism in \ref{hypothesis:1.5}
  between $\pi_1(X,\alpha(0))$ and $\pi_0(X,\alpha(1))$ is
  identified via \cref{eq:1.6.1} with
  the isomorphism \ref{hypothesis:1.2} between
  ${\Pi_1(X)}_{\alpha(0)}$ and ${\Pi_1(X)}_{\alpha(1)}$.

  Moreover, if $X$ is connected with base point $x_0$, the sheaf $\Pi_1(X)$
  corresponds, via equivalence \ref{prop:equivalence_monodromy},
  to the group $\pi_1(X,x_0)$
  with its action on itself by internal automorphisms.
\end{proposition}


Instead of proving the proposition directly, I prefer to relate this definition
with the more common one.

\begin{subpara}
  For $X$ a topological space, its \itblue{fundamental groupoid} is the category
  $\Pi_1(X)$ whose objects are points of $X$ and whose morphisms are homotopy
  classes of paths.

  It turns out that $\Pi_1(X)$ is a \emph{topological groupoid}.
  Let $\widetilde{X}$ be the set of all morphisms in $\Pi_1(X)$.
  Then there is a topology on $\widetilde{X}$ such that all the operations
  (source, target, identity and composition) are continuous maps.

  In this way, we get a covering space
  \[
  \widetilde{X}\markar{(s,t)}X\times X,
  \]
  where $s$ and $t$ are the source and target operations.
\end{subpara}
\begin{Remark}
  In detail, the topology on $\widetilde{X}$ is given as follows.
  First, given any point $x\in X$,
  there is a basis of neighborhoods $\Bb_x$ whose members are path-connected
  open neighborhoods $U\subset X$ of $x$ such that the induced homomorphism
  $\pi_1(U,x)\to\pi_1(X,x)$ is trivial. Then, each of such open neighborhood
  $U$ can be lifted into a subset $\widetilde{U}_x$ of $\widetilde{X}$:
  \[
  \widetilde{U}_x:=
  \big\{[\alpha]:\text{$\alpha$ is a path in $U$ starting from $x$}\big\}.
  \]
  For each point $y$ in $\widetilde{X}$ with a path $\gamma$ presenting it,
  the family
  \[
  \Big\{\widetilde{V}_{\gamma(1)}\gamma\widetilde{U}_{\gamma(0)}^{-1}\Big\}
  _{U\in\Bb_{\gamma(0)},V\in\Bb_{\gamma(1)}},
  \]
  where
  \[
  \widetilde{V}_{\gamma(1)}\gamma\widetilde{U}_{\gamma(0)}^{-1}=
  \big\{
  [\beta\gamma\alpha^{-1}]:
  \alpha\in\widetilde{U}_{\gamma(0)},
  \beta\in\widetilde{V}_{\gamma(1)}
  \big\},
  \]
  form a basis of neighborhoods of $y$ in $\widetilde{X}$.
\end{Remark}

\begin{subpara}
  Let $\Delta\colon X\to X\times X$ be the diagonal map. Then the pullback of
  the covering $\widetilde{X}\to X\times X$ along $\Delta$ gives a covering
  space $\widetilde{X}_{\Delta}\to X$.

  For each point $x$ of $X$, the fiber of $\widetilde{X}_{\Delta}$ at $x$ is
  $\pi_1(X,x)$, and the monodromy actions are given by the inner automorphisms.
  Hence, $\widetilde{X}_{\Delta}$ is a group bundle over $X$.

  Let $\alpha$ be a path in $X$ and $\beta$ a loop presenting a point $y$ in
  the fiber $\widetilde{X}_{\Delta,\alpha(0)}$. Then, the unique lifting of
  $\alpha$ starting from $y$ is given by
  $\widetilde{\alpha}(t)=\alpha_t \beta \alpha_t^{-1}$,
  where $\alpha_t$ is the path $\alpha_t(s):=\alpha(ts)$.
  Therefore, the bijective map
  $\widetilde{X}_{\Delta,\alpha(0)}\to\widetilde{X}_{\Delta,\alpha(1)}$
  given by such liftings is nothing but the conjugate operation
  $[\beta]\mapsto[\alpha][\beta][\alpha]^{-1}$.

  Now, consider the sheaf of sections of $\widetilde{X}_{\Delta}\to X$, denoted
  also by $\Pi_1(X)$. The observations in previous paragraphs implies
  \cref{prop:fundamental_groupoid}.
\end{subpara}

\begin{proposition}
  If $\Ff$ is a locally constant sheaf on $X$, there exists a canonical action
  of $\Pi_1(X)$ on $\Ff$ which, at each $x_0\in X$, induces the action
  \ref{hypothesis:1.2} of $\pi_1(X,x_0)$ on $\Ff_{x_0}$.
\end{proposition}

It suffices to give the action of the group bundle $\widetilde{X}_{\Delta}$ on
an arbitrary covering space $Y\to X$.
This action is precisely the monodromy action, which is already defined in
\ref{def:monodromy_action}.













\clearpage
\section{Integrable connections and local systems}
From now on, an \itblue{analytic space} means a complex analytic space locally
of finite dimension and supposed to be $\sigma$-compact, while not necessarily
separated; a \itblue{complex analytic manifold} means a non-singular (or smooth)
analytic space.

\begin{para}
  Let $X$ be an analytic space. A \itblue{(holomorphic) vector bundle} on $X$ is
  a locally free $\Oo_X$-module of finite type. If $\Vv$ is a vector bundle on
  $X$ and $x$ is a point of $X$,
  we denote by $\Vv_{(x)}$ the finite free $\Oo_{(x)}$-module
  of germs of sections of $\Vv$. If $\mm_x$ is the maximal ideal of $\Oo_{(x)}$,
  the \itblue{fiber} at $x$ of the vector bundle $\Vv$ is
  the following vector space of finite rank:
  \[
  \Vv_x:=\Vv_{(x)}\otimes_{\Oo_{(x)}}\Oo_{(x)}/\mm_x.
  \]

  If $f\colon X\to Y$ is a morphism of analytic spaces,
  the \itblue{pullback} $f^\ast\Vv$ on $X$
  of the vector bundle $\Vv$ on $Y$ is
  \[
  f^\ast\Vv:=\Oo_X\otimes_{f^{-1}\Oo_Y}f^{-1}\Vv.
  \]

  In particular, if $x\colon\pt\to X$ is morphism from the punctual sapce $\pt$
  to $X$ defined by the point $x$ of $X$, we have
  \[
  \Vv_x\cong x^\ast\Vv.
  \]
\end{para}

\begin{para}
  Let $X$ be a complex analytic manifold and $\Vv$ be a vector bundle on $X$.
  The ancients would have defined \emph{ (holomorphic) connection} on $\Vv$
  as the data: for any pair $(x,y)$ of infinitesimally near points of
  first order of $X$, an isomorphism $\gamma_{x,y}\colon\Vv_x\to\Vv_y$,
  this isomorphism depends holomorphically on $(x,y)$ and satisfies
  $\gamma_{x,x}=\Id$.

  If interpreted correctly, this ``definition'' coincides with the definition in
  2.2.4 below (which will not be used in the rest of the section).

  To obtain it, it suffices to interpret ``point'' as
  ``point valued in any analytic space'':
  \begin{subpara}
    A \itblue{point of the analytic space $X$ with values in the analytic space
    $S$} is a morphism from $S$ to $X$.
  \end{subpara}
  \begin{subpara}
    If $Y$ is a subspace of $X$, the \itblue{$n$-th infinitesimal neighborhood}
    of $Y$ in $X$ is the subspace of $X$ locally defined by the $(n+1)$-th power
    of the ideal of $\Oo_X$ defining $Y$.
  \end{subpara}
  \begin{subpara}
    Two points $x,y$ of $X$ with values in $S$ is said to be
    \itblue{infinitesimally near of first order} if
    the map $(x,y)\colon S\to X\times X$ they defined factors through the
    first infinitesimal neighborhood of the diagonal of $X\times X$.
  \end{subpara}
  \begin{subpara}\label{defn:(holomorphic)_connection}
    If $X$ is a complex analytic manifold and $\Vv$ is a vector bundle on $X$,
    a \itblue{(holomorphic) connection} $\gamma$ on $\Vv$ consists of
    the following data:
    \begin{itemize}
      \item for every pair $(x,y)$ of points of $X$ with values in any analytic
      space $S$, with $x$ and $y$ being infinitesimally near of first order,
      we give $\gamma_{x,y}\colon x^\ast\Vv\to y^\ast\Vv$;
    \end{itemize}
    this data is subject to the conditions:
    \begin{enumerate}
      \item For any $f\colon T\to S$ and two points $x,y\colon S\tto X$
      infinitesimally near of first order, we have
      $f^\ast(\gamma_{x,y})=\gamma_{xf,yf}$.
      \item We have $\gamma_{x,x}=\Id$.
    \end{enumerate}
  \end{subpara}
\end{para}

\begin{para}
  Let $X_1$ be the first infinitesimal neighborhood of the diagonal $X_0$ of
  $X\times X$, and $p_1$, $p_2$ the two projections of $X_1$ to $X$.
  By definition, the vector bundle $P^1(\Vv)$ of the \itblue{jets of sections
  of first order} of $\Vv$ is the bundle $p_{1\ast}p_2^{\ast}\Vv$.
  We denote by $j^1$ the differential operator of first order which associates
  each section of $\Vv$ its jet of first order:
  \[
  j^1\colon\Vv\To P^1(\Vv)\cong\Oo_{X_1}\otimes_{\Oo_X}\Vv.
  \]

  A connection in the sense of \ref{defn:(holomorphic)_connection} can be
  interpreted as a homomorphism (automatically isomorphism)
  \[
  \gamma\colon p_1^{\ast}\Vv\To p_2^{\ast}\Vv
  \]
  which induces the identity above $X_0$. Since
  \[
  \Hom_{X_1}(p_1^{\ast}\Vv,p_2^{\ast}\Vv) \cong
  \Hom(\Vv,p_{1\ast}p_2^{\ast}\Vv),
  \]
  a connection is also interpreted as a ($\Oo$-linear) homomorphism
  \[
  D\colon\Vv\To P^1(\Vv)
  \]
  such that the following composition
  \[
  \Vv\markar{D}P^1(\Vv)\To\Vv
  \]
  is the identity.
  The sections $D(s)$ and $j^1(s)$ of $P^1(\Vv)$ have the same image in $\Vv$,
  and $j^1(s)-D(s)$ identifies with a section $\nabla s$ of
  $\Omega_X^1\otimes\Vv\cong\Ker(P^1(\Vv)\to\Vv)$:
  \[
  \nabla\colon\Vv\To\Omega_X^1(\Vv).
  \]
  In other words, a connection (\ref{defn:(holomorphic)_connection}),
  permiting to compare two neighboring fibers of $\Vv$,
  also permit to define the \itblue{differential} $\nabla s$ of a section $s$ of
  $\Vv$.

  Conversely, the formula
  \begin{equation}
    j^1(s) = D(s) + \nabla s
  \end{equation}
  permit to define $D$ hence $\gamma$ from the covariant derivative $\nabla$.
  For $D$ to be linear, it is necessary and sufficient that $\nabla$ satisfies
  the identity
  \begin{equation}\label{eq:Leibniz}
    \nabla(fs) = \di f\otimes s + f.\nabla s.
  \end{equation}
  The definition \ref{defn:(holomorphic)_connection} is therefore equivalent to
  the following definition, due to J.L. Koszul.
\end{para}
\begin{definition}\label{defn:holomorphic_connection}
  Let $\Vv$ be a (holomorphic) vector bundle on a complex analytic manifold $X$.
  A \itblue{holomorphic connection} (or simply, \itblue{connection}) on $\Vv$ is
  $\CC$-linear homomorphism
  \[
  \nabla\colon\Vv\To\Omega_X^1(\Vv):=\Omega_X^1\otimes\Vv
  \]
  satisfying the Leibniz identity \cref{eq:Leibniz} for $f$ and $s$ any local
  sections of $\Oo$ and $\Vv$.
  We call $\nabla$ the \itblue{covariant derivative} defined by the connection.
\end{definition}

\begin{para}
  If the vector bundle $\Vv$ is provided with a connection $\Gamma$ of covariant
  derivative $\nabla$, and if $w$ is a holomorphic vector field on $X$, we put,
  for every local section $v$ of $\Vv$ on open $U$ of $X$,
  \[
  \nabla_w(v):=\<\nabla v, w\> \in\Vv(U).
  \]
  We call $\nabla_w\colon\Vv\to\Vv$ the
  \itblue{covariant derivative along the vector field} $w$.
\end{para}

\begin{para}\label{remark:torsor}
  If $\prescript{}{1}{\Gamma}$ and $\prescript{}{2}{\Gamma}$ are two
  connections, of covariant derivatives $\prescript{}{1}{\nabla}$ and
  $\prescript{}{2}{\nabla}$, then
  $\prescript{}{2}{\nabla}-\prescript{}{1}{\nabla}$ is an $\Oo$-linear
  homomorphism from $\Vv$ to $\Omega_X^1(\Vv)$.
  Conversely, the sum of $\prescript{}{1}{\nabla}$ and such a homomorphism
  defines a connection on $\Vv$.
  Thus, the connections on $\Vv$ form a \emph{homogeneous principal space}
  (or \emph{torsor}) under
  $\CHom(\Vv,\Omega_X^1(\Vv))\cong\Omega_X^1(\CEnd(\Vv))$.
\end{para}

\begin{para}
  If vector bundles are provided with connections, any vector bundle which is
  deduced by a tensor operation is still provided with a connection.
  This is evident on \ref{defn:(holomorphic)_connection}.
  Specifically, let $\Vv_1$ and $\Vv_2$ be two vector bundles with
  connections of covariant derivatives $\prescript{}{1}{\nabla}$ and
  $\prescript{}{2}{\nabla}$.
  \begin{subpara}
    We define a connection on $\Vv_1\oplus\Vv_2$ by the formula
    \[
    \nabla_w(v_1+v_2)=
    \prescript{}{1}{\nabla}_w(v_1)+\prescript{}{2}{\nabla}_w(v_2).
    \]
  \end{subpara}
  \begin{subpara}
    We define a connection on $\Vv_1\otimes\Vv_2$ by the formula
    \[
    \nabla_w(v_1\otimes v_2)=
    \prescript{}{1}{\nabla}_w(v_1)\otimes v_2+
    v_1\otimes\prescript{}{2}{\nabla}_w(v_2).
    \]
  \end{subpara}
  \begin{subpara}\label{defn:connection_on_Hom}
    We define a connection on $\CHom(\Vv_1,\Vv_2)$ by the formula
    \[
    (\nabla_wf)(v_1)=
    \prescript{}{2}{\nabla}_w(f(v_1))-
    f(\prescript{}{1}{\nabla}_w(v_1)).
    \]
  \end{subpara}
  The canonical connection on $\Oo$ is the connection
  for which $\partial f = \di f$.
  Let $\Vv$ be a vector bundle with a connection.
  \begin{subpara}
    We define a connection on the dual $\Vv^{\vee}$ of $\Vv$ via
    \ref{defn:connection_on_Hom} and the isomorphism of the definition
    $\Vv^{\vee}=\CHom(\Vv,\Oo)$. We have
    \[
    \<\nabla_w v', v\> = \partial_w\<v',v\> - \<v',\nabla_w v\>.
    \]
  \end{subpara}
\end{para}

\begin{para}
  An $\Oo$-homomorphism $f$ between vector bundles $\Vv_1$ and $\Vv_2$ equipped
  with connections is said \itblue{compatible with connections} if
  \[
  \prescript{}{2}{\nabla}.f = f.\prescript{}{1}{\nabla}.
  \]
  According to \ref{defn:connection_on_Hom}, this is to say that $\nabla f =0$,
  if $f$ is regarded as a section of $\CHom(\Vv_1,\Vv_2)$.
  For example, according to \ref{defn:connection_on_Hom},
  the canonical homomorphism
  \[
  \CHom(\Vv_1,\Vv_2)\otimes\Vv_1\To\Vv_2
  \]
  is compatible with connections.
\end{para}

\begin{para}
  A local section $v$ of $\Vv$ is said to be \itblue{horizontal} if
  $\nabla v=0$. If $f$ is a homomorphism between vector bundles $\Vv_1$ and
  $\Vv_2$ equipped with connections, it is then the same to say that $f$ is
  horizontal and that $f$ is compatible with connections.
\end{para}

\begin{para}\label{hypothesis:2.10}
  Let $\Vv$ be a holomorphic vector bundle on $X$.
  We put $\Omega_X^p=\bigwedge^p\Omega_X^1$ and
  $\Omega_X^p(\Vv)=\Omega_X^p\otimes_\Oo\Vv$
  (sheaf of the \itblue{external differential $p$-forms with values in} $\Vv$).
  Suppose that $\Vv$ is provided with a holomorphic connection.
  We thus define a $\CC$-linear homomorphism
  \begin{equation}\label{eq:nabla}
    \nabla\colon\Omega_X^p(\Vv)\To\Omega_X^{p+1}(\Vv)
  \end{equation}
  characterized by the following formula
  \begin{equation}\label{eq:Leibniz_higher}
    \nabla(\alpha\otimes v) = \di\alpha\otimes v + (-1)^p\alpha\wedge\nabla v,
  \end{equation}
  where $\alpha$ is a local section of $\Omega_X^p$, $v$ is a local section of
  $\Vv$ and $\di$ is the external differential.
  To verify that the right hand side $\mathrm{I\!I}(\alpha, v)$ of
  \cref{eq:Leibniz_higher} defines a homomorphism \cref{eq:nabla},
  it is sufficient to verify that $\mathrm{I\!I}(\alpha, v)$ is
  $\CC$-bilinear and that
  \[
  \mathrm{I\!I}(f\alpha, v) = \mathrm{I\!I}(\alpha, fv).
  \]
  In fact, we have
  \begin{align*}
    \mathrm{I\!I}(f\alpha, v) &=
    \di(f\alpha)\otimes v + (-1)^pf\alpha\wedge\nabla v \\
    &= \di\alpha\otimes fv + \di f\wedge\alpha\otimes v
    + (-1)^pf\alpha\wedge\nabla v \\
    &= \di\alpha\otimes fv + (-1)^p\alpha\wedge(f\nabla v + \di f\otimes v) \\
    &= \mathrm{I\!I}(\alpha, fv).
  \end{align*}

  Let $\Vv_1$ and $\Vv_2$ be two vector bundles with connections and let $\Vv$
  be their tensor product. We denote by $\wedge$ the evident morphism
  \[
  \wedge\colon\Omega_X^p(\Vv_1)\otimes\Omega_X^q(\Vv_2)\To\Omega_X^{p+q}(\Vv)
  \]
  such that, for $\alpha$, $\beta$, $v_1$, $v_2$ local sections of
  $\Omega_X^p$, $\Omega_X^q$, $\Vv_1$, $\Vv_2$, we have
  \[
  (\alpha\otimes v_1)\wedge(\beta\otimes v_2)
  = (\alpha\wedge\beta)\otimes(v_1\otimes v_2).
  \]
  If $\nu_1$ (resp. $\nu_2$) is a local section of
  $\Omega_X^p(\Vv_1)$ (resp. $\Omega_X^q(\Vv_2)$), we have
  \begin{equation}
    \nabla(\nu_1\wedge\nu_2) =
    \nabla\nu_1\wedge\nu_2 + (-1)^p\nu_1\wedge\nabla\nu_2.
  \end{equation}
  In fact, if $\nu_1=\alpha\otimes v_1$ and $\nu_2=\beta\otimes v_2$, we have
  \begin{align*}
    \nabla(\nu_1\wedge\nu_2) &=
    \nabla(\alpha\wedge\beta\otimes v_1\otimes v_2) \\ &=
    \di(\alpha\wedge\beta)\otimes v_1\otimes v_2 +
    (-1)^{p+q}\alpha\wedge\beta\wedge\nabla(v_1\otimes v_2) \\ &=
    \di\alpha\wedge\beta\otimes v_1\otimes v_2 +
    (-1)^p\alpha\wedge\di\beta\otimes v_1\otimes v_2 \\ &\quad+
    (-1)^{p+q}\alpha\wedge\beta\wedge\nabla v_1 \otimes v_2 +
    (-1)^{p+q}\alpha\wedge\beta\otimes v_1\wedge\nabla v_2 \\ &=
    \di\alpha\otimes v_1\wedge \nu_2 +
    (-1)^p\nu_1\wedge\di\beta\otimes v_2 \\ &\quad+
    (-1)^p\alpha\wedge\nabla v_1\wedge\nu_2 +
    (-1)^{p+q}\nu_1\wedge\beta\wedge\nabla v_2 \\ &=
    \nabla\nu_1\wedge\nu_2 + (-1)^p\nu_1\wedge\nabla\nu_2.
  \end{align*}

  Let $\Vv$ be a vector bundle with a connection.
  If we apply the previous formula to $\Oo$ and $\Vv$, we find that for every
  local section $\alpha$ (resp, $\nu$) of
  $\Omega_X^p$ (resp. $\Omega_X^q(\Vv)$), we have
  \begin{equation}\label{eq:Leibniz_OV}
    \nabla(\alpha\wedge\nu) = \di\alpha\wedge\nu + (-1)^p\alpha\wedge\nabla\nu.
  \end{equation}
  Repeating this formula provides
  \begin{align*}
    \nabla\nabla(\alpha\wedge\nu) &=
    \nabla(\di\alpha\wedge\nu+(-1)^p\alpha\wedge\nabla\nu) \\ &=
    \di\di\alpha\wedge\nu+(-1)^{p+1}\di\alpha\wedge\nabla\nu +
    (-1)^p\di\alpha\wedge\nabla\nu+\alpha\wedge\nabla\nabla\nu \\ &=
    \alpha\wedge\nabla\nabla\nu.
  \end{align*}
\end{para}

\begin{definition}\label{defn:curvature-explicit}
  Under the hypothesis \ref{hypothesis:2.10}, the \itblue{curvature} $\Rr$ of
  the connection on $\Vv$ is composed homomorphism:
  \[
  \nabla\circ\nabla\colon\Vv\To\Omega_X^2(\Vv)
  \]
  viewed as a section of
  $\CHom(\Vv,\Omega_X^2(\Vv))\cong\Omega_X^2(\CEnd(\Vv))$.
\end{definition}

\begin{para}
  The formula \cref{eq:Leibniz_OV} for $q=0$ provides
  \begin{equation}
    \nabla\nabla(\alpha\otimes v) = \alpha\wedge\Rr(v),
  \end{equation}
  which is also written as
  \begin{equation}\label{eq:Ricci}
    \nabla\nabla(\nu)=\Rr\wedge\nu\qquad\text{(\itblue{Ricci identity})}.
  \end{equation}

  Providing $\CEnd(\Vv)$ with the connection \ref{defn:connection_on_Hom},
  the formula $\nabla(\nabla\nabla)=(\nabla\nabla)\nabla$ can be written as
  $\nabla(\Rr\wedge\nu)=\Rr\wedge\nabla\nu$.
  According to \ref{defn:connection_on_Hom}, we have
  $\nabla\Rr\wedge\nu = \nabla(\Rr\wedge\nu) - \Rr\wedge\nabla\nu$ so that
  \begin{equation}\label{eq:Bianchi}
    \nabla\Rr = 0\qquad\text{(\itblue{Bianchi identity})}.
  \end{equation}
\end{para}

\begin{para}\label{defn:curvature}
  If $\alpha$ is a $p$-form, we know that
  \begin{align*}
    \<\di\alpha,X_0\wedge\cdots\wedge X_p\> &=
    \sum (-1)^i\partial_{X_1}
    \<\alpha,X_0\wedge\cdots\widehat{X_i}\cdots\wedge X_p\> \\
    &\quad+
    \sum_{i<j}(-1)^{i+j}\<\alpha,[X_i,X_j]\wedge X_0
    \wedge\cdots\widehat{X_i}\cdots\widehat{X_j}\cdots\wedge X_p\>.
  \end{align*}
  From this formula and \cref{eq:Leibniz_higher}, we find that for every local
  section $\nu$ of $\Omega_X^p(\Vv)$ and holomorphic vector fields
  $X_0,\cdots,X_p$, we have
  \begin{align*}
    \<\nabla\nu,X_0\wedge\cdots\wedge X_p\> &=
    \sum (-1)^i\nabla_{X_1}
    \<\nu,X_0\wedge\cdots\widehat{X_i}\cdots\wedge X_p\> \\
    &\quad+
    \sum_{i<j}(-1)^{i+j}\<\nu,[X_i,X_j]\wedge X_0
    \wedge\cdots\widehat{X_i}\cdots\widehat{X_j}\cdots\wedge X_p\>.
  \end{align*}

  In particular, for $v$ a loacl section of $\Vv$, we have
  \[
  \<\nabla\nabla v, X_1\wedge X_2\> =
  \nabla_{X_1}\<\nabla v,X_2\> - \nabla_{X_2}\<\nabla v,X_1\> +
  \<\nabla,[X_1,X_2]\>.
  \]
  Let
  \[
  \Rr(X_1,X_2)(v):=
  \nabla_{X_1}\nabla_{X_2}v-\nabla_{X_2}\nabla_{X_1}v
  -\nabla_{[X_1,X_2]}v.
  \]
\end{para}

\begin{definition}\label{defn:integrable_connections}
  A connection is said to be \itblue{integrable} if its curvature is zero, i.e.
  if we have tha identity
  \[
  \nabla_{[X,Y]} = [\nabla_X,\nabla_Y].
  \]
\end{definition}
If $\dim(X)\le 1$, every connection is integrable.

If $\Gamma$ is a integrable connection on $\Vv$, the morphism $\nabla$ in
\cref{eq:nabla} satisfies $\nabla\nabla=0$, so that $\Omega_X^p(\Vv)$ forms
a differential complex $\Omega_X^\bullet(\Vv)$.
\begin{definition}\label{defn:DeRhamComplex}
  Under the previous hypothesis, the complex $\Omega_X^\bullet(\Vv)$ is called
  the \itblue{holomorphic De Rham complex} with values in $\Vv$.
\end{definition}

The following results
\ref{prop:canonical_connection}--\ref{prop:Poincare_lemma}
will be presented more generally in \ref{thm:Riemann–Hilbert correspondence}.

\begin{proposition}\label{prop:canonical_connection}
  Let $V$ be a complex local system on a complex analytic manifold $X$ and
  $\Vv=\Oo\otimes_\CC V$.
  \begin{proplist}
    \item There exists a canonical connection on $\Vv$, for which the horizontal
    sections of $\Vv$ are local sections of the subsheaf $V$ of $\Vv$.
    \item The canonical connection on $\Vv$ is integrable.
    \item For $f$ (resp. $v$) a local section of $\Oo$ (resp. $V$), we have
    \begin{equation}\label{eq:horizontal}
      \nabla(fv)=\di f\otimes v.
    \end{equation}
  \end{proplist}
\end{proposition}

If $\nabla$ satisfies (i), then \cref{eq:horizontal} is just a special case
of \cref{eq:Leibniz}. Conversely, the right hand side of \cref{eq:horizontal}
is $\CC$-bilinear and extends uniquely to a $\CC$-linear homomorphism
$\nabla\colon\Vv\to\Omega_X^1(\Vv)$, which satisfies the definition of a
connection. The assumption (ii) is local on $X$, which permits to reduce to the
case $V=\underline{\CC}$. At the moment, $\Vv=\Oo$, $\nabla=\di$ and
$\nabla_{[X,Y]}=[\nabla_X,\nabla_Y]$ by definition of $[X,Y]$.

It is well known that
\begin{theorem}\label{thm:canonical_connection}
  Let $X$ be a complex analytic manifold, the following functors
  \begin{enumerate}[a)]
    \item sending a complex local system $V$ to the vector bundle
    $\Vv=\Oo\otimes_\CC V$ equipped with the canonical connection,
    \item sending a holomorphic vector bundle $\Vv$ on $X$, equipped with an
    integrable connection, to the subsheaf $V$ of horizontal sections of $\Vv$,
  \end{enumerate}
  form a pair of equivalences between
  the category of complex local systems on $X$ and
  the category of holomorphic vector bundles with integrable connections on $X$
  (with morphisms the horizontal morphisms of vector bundles).
\end{theorem}

These equivalences are compatible with the formation of the tensor product, the
internal Hom and the dual. The unit complex local system $\underline{\CC}$
corresponds to the bundle $\Oo$, equipped with the connection such that
$\nabla f = \di f$.

One deduces from \cref{eq:Leibniz_higher} that
\begin{proposition}\label{prop:Isom_DeRhamComplex}
  If $V$ is a complex local system on $X$, and if $\Vv=\Oo\otimes_\CC V$,
  then the system of isomorphisms $\Omega_X^p\otimes_\CC V \cong
  \Omega_X^p\otimes_\Oo\Oo\otimes_\CC V = \Omega_X\otimes_\Oo\Vv$ is an
  isomorphism of complexes
  \[
  \Omega_X^\bullet\otimes_\CC V \Isom \Omega_X^\bullet(\Vv).
  \]
\end{proposition}

From this, the \emph{holomorphic Poincar\'{e} lemma} results that
\begin{proposition}\label{prop:Poincare_lemma}
  Under the hypothesis of \ref{prop:canonical_connection},
  the complex $\Omega_X^\bullet(\Vv)$ is a resolution of the sheaf $V$.
\end{proposition}

\begin{para}
  Variants.
  \begin{subpara}
    If $X$ is a differential manifold, for the $C^\infty$-connections on the
    $C^\infty$-vector bundles, all the above results remains valid,
    \emph{mutatis mutandis}.
  \end{subpara}
  \begin{subpara}
    \Cref{thm:canonical_connection} essentially requires the non-singularity of
    $X$; it is therefore of no interest to note that this hypothesis was not
    used in an essential way before \ref{thm:canonical_connection}.
  \end{subpara}
  \begin{subpara}
    The \cref{defn:holomorphic_connection} of a connection and
    the \cref{defn:integrable_connections} of integrablity are
    sufficiently formal to be transposed in the category of schemes, or in
    related situations.
  \end{subpara}
\end{para}

\begin{definition}\label{defn:connections_on_schemes}
  \begin{paras}
    \item Let $f\colon X\to S$ be a smooth morphism of schemes and $\Vv$ a
    quasi-coherent sheaf on $X$. A \itblue{relative connection} on $\Vv$ is an
    $f^{-1}\Oo_S$-linear morphism of sheaves
    (called the \itblue{covariant derivative} defined by the connection)
    \[
    \nabla\colon\Vv\To\Omega_{X/S}^1(\Vv)
    \]
    satisfying following identity, for $f$ (resp. $v$) a local section of
    $\Oo_X$ (resp. $\Vv$),
    \[
    \nabla(fv) = \di f\otimes v + f\nabla v.
    \]
    \item For $\Vv$ equipped with a relative connection, there exists a unique
    system of $f^{-1}\Oo_S$-homomorphisms of sheaves
    \[
    \nabla^{(p)}\text{ or }\nabla\colon
    \Omega_{X/S}^p(\Vv)\To\Omega_{X/S}^{p+1}(\Vv)
    \]
    satisfying the identity \cref{eq:Leibniz_OV} and such that
    $\nabla^{(0)}=\nabla$.
    \item The \itblue{curvature} of a connection is defined by
    \[
    \Rr=\nabla^{(1)}\circ\nabla^{(0)} \in
    \CHom(\Vv,\Omega_{X/S}^2(\Vv))\cong\Omega_{X/S}^2(\CEnd{\Vv}).
    \]
    The curvature satisfies the Ricci identity \cref{eq:Ricci} and the Bianchi
    identity \cref{eq:Bianchi}.
    \item An \itblue{integrable connection} is a connection with zero curvature.
    \item The \itblue{De Rham complex} defined by an integrable connection is
    the complex $(\Omega_{X/S}^\bullet(\Vv),\nabla)$.
  \end{paras}
\end{definition}

\begin{para}\label{defn:relative_connections}
  Let $f\colon X\to S$ be a \itblue{smooth} morphism of complex analytic spaces,
  that means, locally on $X$, it is isomorphic to the projection from
  $D^n\times S$ to $S$, where $D^n$ is an open polydisc.
  A \itblue{relative local system} on $X$ is an $f^{-1}\Oo_S$-module, locally
  isomorphic to a pullback of a coherent analytic sheaf on $S$.
  If $\Vv$ is a coherent analytic sheaf on $X$, a \itblue{relative connection}
  on $\Vv$ is an $f^{-1}\Oo_S$-linear homomorphism
  \[
  \nabla\colon\Vv\To\Omega_{X/S}^1(\Vv)
  \]
  satisfying following identity, for $f$ (resp. $v$) a local section of
  $\Oo_X$ (resp. $\Vv$),
  \[
  \nabla(fv) = \di f\otimes v + f\nabla v.
  \]
  A \itblue{(horizontal) morphism} between vector bundles with relative
  connections is a morphism bewteen vector bundles commuting with $\nabla$.
  We define as in \ref{defn:curvature} and \ref{defn:connections_on_schemes}
  the \itblue{curvature} $\Rr\in\Omega_{X/S}^2(\CEnd(\Vv))$ of a relative
  connection. A connection is said to be \itblue{integrable} if $\Rr=0$,
  in which case we have the \itblue{relative De Rham complex}
  $\Omega_{X/S}^\bullet(\Vv)$ with values in $\Vv$, defined as in
  \ref{defn:DeRhamComplex} and \ref{defn:connections_on_schemes}.
\end{para}

The ``absolute'' statements
\ref{thm:canonical_connection}, \ref{prop:Isom_DeRhamComplex} and
\ref{prop:Poincare_lemma} have ``relative'' (i.e. ``with parameters'')
analogies.

\begin{theorem}\label{thm:Riemann–Hilbert correspondence}
  Under the hypothesis of \ref{defn:relative_connections}, we have
  \begin{paras}
    \item For every relative local system $V$ on $X$, there exists a coherent
    analytic sheaf $\Vv=\Oo_X\otimes_{f^{-1}\Oo_S}V$ with a canonical relative
    connection, such that a local section $v$ of $\Vv$ is horizontal
    ($\nabla v=0$) if and only if $v$ is a section of $V$.
    Moreover, this connection is integrable.
    \item Given a relative local system $V$ on $X$,
    the De Rham complex defined by $\Vv=\Oo_X\otimes_{f^{-1}\Oo_S}V$,
    equipped with its canonical connection,
    is a resolution of the sheaf $V$.
    \item The following functors
    \begin{enumerate}[a)]
      \item sending a relative local system $V$ to the coherent analytic sheaf
      $\Vv=\Oo_X\otimes_{f^{-1}\Oo_S}V$ equipped with the canonical connection,
      \item sending a coherent analytic sheaf $\Vv$ on $X$, equipped with an
      integrable relative connection, to the subsheaf of horizontal sections
      of $\Vv$,
    \end{enumerate}
    form a pair of equivalences between
    the category of relative local systems on $X$ and
    the category of coherent analytic sheaves on $X$, equipped with integrable
    relative connections.
  \end{paras}
\end{theorem}

Proof of (i).
To verify that $\Vv$ is coherent, it suffices to do it locally,
for $V=f^{-1}V_0$, in which case $\Vv$ is the pullback,
in the sense of coherent analytical sheaves, of $V_0$.
The canonical relative connection necessarily verifies, for $f$ (resp. $v$)
a loal section of $\Oo$ (resp. $\Vv$),
\begin{equation}\label{eq:horizontal_relative}
  \nabla(fv)=\di f\otimes v.
\end{equation}
The right hand side $\mathrm{I\!I}(f,v)$ is biadditive in $f$ and $v$, and
satisfies, for $g$ a local section of $f^{-1}\Oo_S$, the identity
\[
\mathrm{I\!I}(fg,v) = \mathrm{I\!I}(f,gv),
\]
(using that $\di g=0$ in $\Omega_{X/S}^1$).
We can deduce the existence and the uniqueness of a relative connection
$\nabla$ satisfying \cref{eq:horizontal_relative}. We finally have
\[
\nabla\nabla(fv)=\nabla(\di f\otimes v)=\di\di f\otimes v = 0;
\]
the canonical connection is thus integrable.
That the sections of $V$ are the only horizontal ones is a special case of (ii)
proving below.

\begin{subpara}
  Let's first look at the particular case of (ii) where $S=D^n$,
  $X=D^n\times D^m$, $f=\pr_2$ and where the relative local system $V$ is the
  pullback of $\Oo_S$.
  The complex of global sections
  \[
  0\To\Gamma(f^{-1}\Oo_S)\To\Gamma(\Oo_X)\markar{\di}
  \Gamma(\Omega_{X/S}^1)\To\cdots
  \]
  is acyclic, because it admits the following homotopy operator.
  \begin{enumerate}[a)]
    \item $H\colon\Gamma(\Oo_X)\to\Gamma(f^{-1}\Oo_S)=\Gamma(S,\Oo_S)$ is the
    pullback by zero section of $f$;
    \item $H\colon\Gamma(\Omega_{X/S}^p)\to\Gamma(\Omega_{X/S}^{p-1})$ is given
    as follows: since $H$ must be $\Gamma(f^{-1}\Oo_S)$-linear
    and $\Omega_{X/S}^p$ has a basis $\{x^{\underline{n}}\di x_I :
    \underline{n}\in\NN^m,I\subset[1,m],|I|=p\}$, it suffices to define
    \[
    H(x^{\underline{n}}\di x_I)=
    \frac{1}{m}\sum_{i\in I}\frac{\sgn_I(i)}
    {n_i+1}x^{\underline{n}+\epsilon_i}\di x_{I\setminus\{i\}},
    \]
    where $\sgn_I(i)$ is the signature of $i$ in the seuqence $I$,
    and $\epsilon_i$ is the $i$-th member of the standard basis of $\NN^m$.
  \end{enumerate}

  This remains true if we replace $D^{m+n}$ by a smaller polycylinder.
  Therefore the complex of sheaves
  \[
  0\To f^{-1}\Oo_S\To \Oo_X\markar{\di} \Omega_{X/S}^1\To\cdots
  \]
  is acyclic and thus $\Omega_{X/S}^\bullet$ is a resolution of $f^{-1}\Oo_S$.
\end{subpara}
\begin{subpara}
  Proof of (ii).
  The assertion (ii) is naturally local on $X$ and $S$. Denoted by $D$ the unit
  open disk, so we can go back to the case where $S$ is a closed analytical
  subset of the polycylinder $D^n$, where $X=D^m\times S$, with $f=\pr_2$, and
  where $V$ is the pullback of a coherent analytic sheaf $V_0$ on $S$.
  Applying the syzygy theorem, and shrinking $X$ and $S$, we can further assume
  that the pushforward of $V_0$ on $D^n$, which also denoted by $V_0$, admits a
  finite resolution $\Ll^\bullet$ by free coherent $\Oo_{D^n}$-modules.
  To prove (ii), it is permissible to replace $V_0$ by its pushforward on $D^n$,
  which will be done henceforth.

  If $\Sigma_0$ is a short exact sequence of coherent $\Oo_S$-moduels
  \[
  \Sigma_0\colon
  0\To V_0'\To V_0\To V_0''\To 0,
  \]
  let $V=f^{-1}V_0$ be the short exact sequence of relative local systems which
  is the pullback of $\Sigma_0$ (the sequence $\Sigma$ is exact because $f^{-1}$
  is an exact functor) and let $\Omega_{X/S}^\bullet(\Sigma)$ be the
  corresponding exact sequence of relative De Rham complexes
  \[
  %\Omega_{X/S}^\bullet(\Sigma)\colon
  0\To \Omega_{X/S}^\bullet\otimes_{f^{-1}\Oo_S}f^{-1}V_0'
  \To \Omega_{X/S}^\bullet\otimes_{f^{-1}\Oo_S}f^{-1}V_0
  \To \Omega_{X/S}^\bullet\otimes_{f^{-1}\Oo_S}f^{-1}V_0''\To 0.
  \]
  This sequence is exact because $\Omega_{X/S}^\bullet$ is flat over
  $f^{-1}\Oo_S$, being locally free on $\Oo_X$ which is flat on $f^{-1}\Oo_S$.

  The snake lemma applied to $\Omega_{X/S}^\bullet(\Sigma)$ shows that if the
  assertion (ii) is true for two of $f^{-1}V_0'$, $f^{-1}V_0$ and $f^{-1}V_0''$,
  then it is also true for the third. We can deduce by induction that if $V_0$
  admits a finite resolution by modules satisfying (ii), then so does $V_0$.
  This, applied to $V_0$ and $\Ll^\bullet$, complete the proof of (ii).
\end{subpara}
\begin{subpara}
  It follows from (ii) that the composition of functors in (iii)
  (in the order $b\circ a$) is canonically isomorphic to the identity;
  in addition, if $V_1$ and $V_2$ are two relative local systems, and
  $u\colon\Vv_1\to\Vv_2$ is a homomorphism inducing $0$ on $V_1$, then $u=0$
  since $V_1$ generates $\Vv_1$; it follows that the functor a is fully faithful.
  It remains to show that any vector bundle $\Vv$ equipped with a integrable
  relative connection $\nabla$ locally comes from a relative local system.

  \paragraph{Case 1.}
  $S=D^n$, $X=D^{n+1}=D^n\times D$, $f=\pr_1$ and $\Vv$ is free.

  Under these assumptions, if $v$ is any section of the pullback of $\Vv$ along
  the zero section $s_0$ of $f$, there exists a unique horizontal section
  $\widetilde{v}$ of $\Vv$ which coincides with $v$ on $s_0(S)$
  (existence and uniqueness from a Cauchy problem with parameters).
  If $(e_i)$ is a basis of $s_0^{\ast}\Vv$, then $\widetilde{e_i}$ form a
  horizontal basis of $\Vv$, and $(\Vv,\nabla)$ is defined by the relative
  local system $f^{-1}s_0^{\ast}\Vv=f^{-1}\Oo_S^k$.

  \paragraph{Case 2.}
  $S=D^n$, $X=D^{n+1}=D^n\times D$ and $f=\pr_1$.

  Shrinking $X$ and $S$, we may assume that $\Vv$ admits a free presentation
  \[
  \Vv_1\markar{d}\Vv_0\markar{\epsilon}\Vv\To0
  \]
  Shrinking further, we reduce to the case where $\Vv_0$ and $\Vv_1$ admit
  connections $\prescript{}{0}{\nabla}$ and $\prescript{}{1}{\nabla}$, such that
  $\epsilon$ and $d$ are compatible with connections
  (if $(e_i)$ is a basis of $\Vv_0$, $\prescript{}{0}{\nabla}$ is determined by
  $\prescript{}{0}{\nabla}e_i$, and it suffices to choose
  $\prescript{}{0}{\nabla}e_i$ such that
  $\epsilon(\prescript{}{0}{\nabla}e_i) = \nabla(\epsilon(e_i))$;
  similarly for $\prescript{}{1}{\nabla}$).
  The connections $\prescript{}{0}{\nabla}$ and $\prescript{}{1}{\nabla}$ are
  automatically integrable, since $f$ is of relative dimension $1$.
  Thus there exist (case 1) relative local systems $V_0$ and $V_1$ such that
  $(\Vv_i,\prescript{}{i}{\nabla})\cong\Oo_X\otimes_{f^{-1}\Oo_S}V_i$.
  We thus have
  \[
  (\Vv,\nabla) \cong \Oo_X\otimes_{f^{-1}\Oo_S}(V_0/dV_1).
  \]

  \paragraph{Case 3.}
  $f$ is of relative dimension $1$.

  We may assume that $S$ is a closed analytical subset of the polycylinder $D^n$
  and that $X=\times D$, $f=\pr_1$. The relative local systems (resp. relative
  moduels with connection) on $X$ then identify with relative local systems
  (resp. relative modules with connection) on $D^n\times D$ annihilated by the
  pullback of the ideal defining $S$, and we conclude by case 2.

  \paragraph{General case.}
  Prove by induction on the relative dimension $n$ of $f$.
  The case $n=0$ is trivial. If $n\neq0$, we can reduce to the case where
  $X=S\times D^{n-1}\times D$ and where $f=\pr_1$.
  The bundle with connection $(\Vv,\nabla)$ induces on
  $X_0=S\times D^{n-1}\times\{ 0 \}$ a bundle with connection $\Vv_0$ which,
  by inductive hypothesis, is of the type $(\Vv_0,\prescript{}{0}{\nabla}) =
  \Oo_{X_0}\otimes_{\pr_1^{-1}\Oo_S}V$. The projection $p$ from $X$ to $X_0$ is
  of relative dimension $1$, and the relative connection $\nabla$ induces a
  relative connection for $\Vv$ on $X\to X_0$.
  According to case 3, there exists a vector bundle $V_1$ on $X_0$ and of an
  isomorphism of bundles with relative connection (for $p$) such that
  \[
  \Vv\cong\Oo_X\otimes_{p^{-1}\Oo_{X_0}}p^{-1}V_1.
  \]

  The vector bundle $V_1$ is identified with the restriction of $\Vv$ on $X_0$,
  hence an isomorphism of vector bundles
  \[
  \alpha\colon\Vv\cong\Oo_X\otimes_{f^{-1}\Oo_S}V
  \]
  satisfying
  \begin{enumerate}[(i)]
    \item The restriction of $\alpha$ to $X_0$ is horizontal.
    \item $\alpha$ is ``relatively horizontal'' for $p$.
  \end{enumerate}

  If $v$ is a section of $V$, then condition (ii) means that
  \[
  \nabla_{x_n}v=0.
  \]
  If $1\le i<n$, and since $\Rr=0$, we have by the relative analogy of
  \ref{defn:curvature-explicit}:
  \[
  \nabla_{x_n}\nabla_{x_i}v = \nabla_{x_i}\nabla_{x_n}v = 0.
  \]
  In other words, $\nabla_{x_i}v$ is a relative horizontal section, for $p$, of
  $\Vv$; according to (i), it vanishes on $X_0$, hence it is zero and we conclude
  that $\nabla v=0$. The isomorphism $\alpha$ is thus horizontal and this finish
  the proof of \ref{thm:Riemann–Hilbert correspondence}.
\end{subpara}

Some general topology results
(\ref{recal:cohomology_paracompact}--\ref{recall:2.27})
will be needed to deduce
\ref{prop:Gauss-Manin} from \ref{thm:Riemann–Hilbert correspondence}.

\begin{recall}\label{recal:cohomology_paracompact}
  Let, in a topological space $X$, $Y$ be a closed subspace having a
  \emph{paracompact} neighborhood. For each sheaf $\Ff$ on $X$, we have
  \[
  \dirlim H^\bullet(U,\Ff)\Isom H^\bullet(Y,\local{\Ff}{Y})
  \]
  where $U$ goes through neighborhoods of $Y$.
\end{recall}

This is Godement {\cite{Godement}*{II 4.11.1}}.
%Indeed, the condition implies that
%\[
%\check{C}^\bullet(Y,\local{\Ff}{Y})=
%\dirlim_{U\supset Y}\check{C}^\bullet(U,\Ff).
%\]

\begin{corollary}\label{cor:stalk_Rf}
  Let $f\colon X\to S$ be a proper and separated map between topological spaces.
  Suppose that $S$ is locally paracompact. Then, for any point $s$ and any sheaf
  $\Ff$ on $X$, we have
  \[
  (R^if_{\ast}\Ff)_s = H^i(f^{-1}(s),\local{\Ff}{f^{-1}(s)}).
  \]
\end{corollary}

Since $f$ is closed, the collection of $f^{-1}(U)$, where $U$ goes through all
neighborhoods of $s$, form a basis of neighborhoods of $f^{-1}(s)$. Moreover,
if $U$ is paracompact, so is $f^{-1}(U)$ because $f$ is proper and separated.
Note that
\[
(R^if_{\ast}\Ff)_s = \dirlim_{s \in U}H^i(f^{-1}(U),\Ff).
\]
We conclude by \ref{recal:cohomology_paracompact}.

\begin{sublemma}
  Let $f\colon X\to S$ be a surjective proper map between topological spaces.
  If $S$ is paracompact, then so is $X$.
\end{sublemma}

For any subset $Y$ of $X$, define
\[
g(Y) = S\setminus f(X\setminus Y).
\]
Then $g$ maps opens of $X$ to opens of $S$ because $f$ is closed.
Let $\{U_i\}_{i\in I}$ be an open cover of $X$. Then since each fiber
$f^{-1}(s)$ is quasi-compact, it admits a finite open cover
$\{U_i\cap f^{-1}(s)\}_{i\in I_s}$, where $I_s$ is a finite subset of $I$.
Then the family
\[
\Big\{ g(\bigcup_{i\in I_s} U_i) : s\in S \Big\}
\]
is an open cover of $S$. Then it admits a locally finite cover
$\{V_j\}_{j\in J}$. Now, for each $V_j\in g(\bigcup_{i\in I_s} U_i)$, let
$W_{j,i}=f^{-1}(V_j)\cap U_i$ ($i\in I_s$). Then $\{W_{j,i}\}_{j\in J,i\in I}$
is a locally finite refinement of $\{U_i\}_{i\in I}$.

\begin{recall}\label{recall:projection}
  Let $X$ be a locally contractible paracompact space, $i$ an integer and $V$ a
  complex local system on $X$, satisfying $\dim_{\CC}H^i(X,V)<\infty$. Then,
  for every vector space $A$ over $\CC$, can be of infinite dimension, qw have
  \begin{equation}\label{eq:2.26.1}
    A\otimes_{\CC}H^i(X,V) \Isom H^i(X,A\otimes_{\CC}V).
  \end{equation}
\end{recall}

Denoted by $H_\bullet(X,V^\ast)$ the singular homology of $X$ with coefficients
in $V^\ast$. The universal coefficient theorem, valid here, give
\begin{equation}\label{eq:2.26.2}
  H^i(X,A\otimes_{\CC}V) \Isom \Hom_{\CC}(H_i(X,V^{\ast}),A).
\end{equation}
For $A=\CC$, we conclude that $\dim_{\CC} H_i(X,V^{\ast})<\infty$.
Then formula \cref{eq:2.26.1} follows from \cref{eq:2.26.2}.

\begin{para}\label{recall:2.27}
  Let $f\colon X\to S$ be a smooth and separated morphism bewteen complex
  analytic spaces and let $V$ be a local system on $X$. The sheaf
  \begin{equation}\label{eq:2.27.1}
    V_{rel}:=f^{-1}\Oo_S\otimes_{\CC}V
  \end{equation}
  is then a relative local system. We denote by $\Omega_{X/S}^\bullet(V)$ the
  corresponding De Rham complex.
  According to \ref{thm:Riemann–Hilbert correspondence}, $\Omega_{X/S}^\bullet$
  is a resolution of $V_{rel}$. We thus have
  \begin{equation}\label{eq:2.27.2}
    R^if_{\ast}(V_{rel}) \Isom R^if_{\ast}(\Omega_{X/S}^\bullet(V))
  \end{equation}
  where right hand side is a relative hypercohomology. From \cref{eq:2.27.1},
  we deduce a projection arrow
  \begin{equation}\label{eq:2.27.3}
    \Oo_S\otimes_{\CC}R^if_{\ast}V\To R^if_{\ast}(V_{rel}),
  \end{equation}
  which comes from the canonical arrow
  \[
  Lf^{-1}(\Oo_S\otimes^L_{\CC}Rf_{\ast}V)=
  Lf^{-1}\Oo_S\otimes^L_{\CC}LfRf_{\ast}V\To
  Lf^{-1}\Oo_S\otimes^L_{\CC}V.
  \]
  From \cref{eq:2.27.3}, by composition with \cref{eq:2.27.2}, we get an arrow
  \begin{equation}\label{eq:2.27.4}
    \Oo_S\otimes_{\CC}R^if_{\ast}V\To R^if_{\ast}(\Omega_{X/S}^\bullet(V)).
  \end{equation}
\end{para}

\begin{proposition}\label{prop:Gauss-Manin}
  Let $f\colon X\to S$ be a smooth and separated morphism bewteen complex
  analytic spaces, $i$ an integer and $V$ a complex local system on $X$.
  We suppose that
  \begin{enumerate}[a)]
    \item locally on $S$, $f$ is topologically trivial;
    \item the fiber of $f$ satisfies
    \[
    \dim_{\CC}H^i(f^{-1}(s),V)<\infty.
    \]
  \end{enumerate}
  Then, the arrow \cref{eq:2.27.4} is an isomorphism
  \[
  \Oo_S\otimes_{\CC}R^if_{\ast}V\Isom R^if_{\ast}(\Omega_{X/S}^\bullet(V)).
  \]
\end{proposition}

Let $s\in S$, $Y=f^{-1}(s)$, and $V_0=\local{V}{Y}$. To verify \cref{eq:2.27.4}
is an isomorphism, it suffices to construct a basis of neighborhoods $T$ of $s$
such that the arrows
\begin{equation}\label{eq:2.28.1}
  H^0(T,\Oo_S)\otimes H^i(T\times Y,\pr_2^{-1}V_0)\Isom
  H^i(T\times Y,\pr_1^{-1}\Oo_S\otimes\pr_2^{-1}V_0)
\end{equation}
are isomorphisms.
If so, then the stalk of \cref{eq:2.27.3} at $s$, as an inductive limit of
\cref{eq:2.28.1} because of locality of cohomology, will also be an isomorphism.

We'll prove this for $T$ a contractible Stein compact neighborhood of $s$.
Note that \cref{eq:2.28.1} can be also written as
\begin{equation}\label{eq:2.28.2}
  H^0(T,\Oo_S)\otimes H^i(Y,V_0)\Isom
  H^i(T\times Y,\pr_1^{-1}\Oo_S\otimes\pr_2^{-1}V_0).
\end{equation}

Calculate the right hand side of \cref{eq:2.28.2} by Leray spectral sequence
for $\pr_2\colon T\times Y\to Y$. First, we have
\[
H^i\big(Y,R^j(\pr_2)_\ast(\pr_1^{-1}\Oo_S\otimes\pr_2^{-1}V_0)\big)
\then
H^{i+j}(T\times Y,\pr_1^{-1}\Oo_S\otimes\pr_2^{-1}V_0).
\]
By projection formula, we have
\[
R^j(\pr_2)_\ast(\pr_1^{-1}\Oo_S\otimes\pr_2^{-1}V_0) \cong
R^j(\pr_2)_\ast\pr_1^{-1}\Oo_S\otimes V_0.
\]
According to \ref{cor:stalk_Rf}, we have
\[
(R^j(\pr_2)_\ast\pr_1^{-1}\Oo_S)_y
\Isom H^j(T,\Oo_S)
\]
for all $y\in Y$, hence it is a constant sheaf.
Since $T$ is Stein, $H^j(T,\Oo_S)=0$ for $j>0$.
Finally, we have
\[
H^i(T\times Y,\pr_1^{-1}\Oo_S\otimes\pr_2^{-1}V_0)
=H^i(Y,H^0(T,\Oo_S)\otimes V_0).
\]
We conclude by \ref{recall:projection}.

\begin{para}
  Under the hypothesis of \ref{prop:Gauss-Manin}, with $S$ smooth, we define
  the \itblue{Gauss-Manin connection} on $R^if_{\ast}(\Omega_{X/S}(V))$ as
  being the unique integrable connection admitting for horizontal local sections
  the local sections of $R^if_{\ast}V$ (\ref{thm:canonical_connection}).
\end{para}











\clearpage
\section{Translation in terms of partial differential equations of first order}
\begin{para}
  Let $X$ be a complex analytic manifold. If $\Vv$ is a holomorphic vector
  bundle defined by a $\CC$-vector space $V_0$, we have seen that $\Vv$ admits
  a canonical connection of covariant derivative $\prescript{}{0}{\nabla}$.
  If $\nabla$ is the covariant derivative defined by another connection on $\Vv$,
  we have seen (\ref{remark:torsor}) that $\nabla$ is written in the form
  \[
  \nabla = \prescript{}{0}{\nabla}+\Gamma,\qquad
  \text{ where }\Gamma\in\Omega(\CEnd(\Vv)).
  \]

  If we identify sections of $\Vv$ with holomorphic maps from $X$ to $V_0$
  (for example, the section $f\otimes v_0$ with $v_0\in V_0$ can be viewed as
  such a map by $x\mapsto f(x)\otimes v_0$),
  then we have
  \begin{equation}\label{eq:3.1.1}
    \nabla v = \di v + \Gamma v.
  \end{equation}

  If we choose a basis of $V$, i.e. an isomorphism $e\colon \CC^n\to V_0$ of
  coordinates (identified with vectors of basis) $e_\alpha\colon\CC\to V_0$,
  then $\Gamma$ can be represented by a matrix of formal differential
  $\omega^\alpha_\beta$ (the \itblue{matrix of forms of connection}), and
  \cref{eq:3.1.1} can be rewritten as
  \begin{equation}\label{eq:3.1.2}
    (\nabla v)^\alpha = \di v^\alpha + \sum_\beta\omega^\alpha_\beta v^\beta.
  \end{equation}

  Let $\Vv$ be any holomorphic vector bundle on $X$. The choice of a basis
  $e\colon\CC^n\to\Vv$ of $\Vv$ permit to consider $\Vv$ as defined by the
  constant vector space $\CC^n$, and the previous considerations apply:
  the connection on $\Vv$ corresponds, via \cref{eq:3.1.2}, to the $n\times n$
  matrix of differential forms on $X$.
  If $\omega_e$ is the matrix of the connection $\nabla$ under the basis $e$,
  and if $f\colon\CC^n\to\Vv$ is a new basis of $\Vv$, of coordinate
  $A\in\GL_n(\Oo)$ ($A=ef^{-1}$), we have, by \cref{eq:3.1.2},
  \begin{align*}
    \nabla v &= e\di(e^{-1}v) + e\omega_ee^{-1}v \\
    &= fA^{-1}\di(Af^{-1}v) + fA^{-1}\omega_eAf^{-1}v \\
    &= f\di(f^{-1}v) + f(A^{-1}\di A + A^{-1}\omega_eA)f^{-1}v.
  \end{align*}
  Comparing with \cref{eq:3.1.2} under the basis $f$, we find that
  \begin{equation}\label{eq:3.1.3}
    \omega_f = A^{-1}\di A + A^{-1}\omega_eA.
  \end{equation}

  Moreover, if $(x^i)$ is a system of local coordinates on $X$, defining a bsis
  of $\Omega_X^1$ with basis vectors $\di x^i$, we put
  \[
  \omega^\alpha_\beta = \sum_{i}\Gamma^\alpha_{\beta i}\di x^i
  \]
  and we call the holomorphic functions $\Gamma^{\alpha}_{\beta i}$
  the \itblue{coefficients of the connection}.
  The formula \cref{eq:3.1.2} can rewritten as
  \begin{equation}
    (\nabla_i v)^\alpha =
    \partial_iv^\alpha + \sum_\beta\Gamma^\alpha_{\beta,i}v^\beta.
  \end{equation}

  The differential equation $\nabla v = 0$ of horizontal sections of $\Vv$ can
  be rewritten as the system of
  homogeneous linear partial differential equations of first order,
  \begin{equation}\label{eq:3.1.5}
    \partial_iv^\alpha = - \sum_\beta\Gamma^\alpha_{\beta,i}v^\beta.
  \end{equation}
\end{para}

\begin{para}
  With the notations of \cref{eq:3.1.2},
  using the summation convention of dummy indexes,
  we have
  \begin{align*}
    \nabla\nabla v &=
    \nabla((\di v^\alpha + \omega^\alpha_\beta v^\beta).e_\alpha) \\&=
    \di(\di v^\alpha + \omega^\alpha_\beta v^\beta).e_\alpha -
    (\di v^\alpha + \omega^\alpha_\beta v^\beta)
    \wedge\omega^\gamma_\alpha.e_\gamma \\&=
    \di\omega^\alpha_\beta.v^\beta.e_\alpha -
    \omega^\alpha_\beta\wedge\di v^\beta.e_\alpha -
    \di v^\alpha\wedge\omega^\gamma_\alpha.e_\gamma -
    \omega^\alpha_\beta\wedge\omega^\gamma_\alpha.v^\beta.e_\gamma \\&=
    (\di\omega^\gamma_\beta-\omega^\alpha_\beta\wedge\omega^\gamma_\alpha)
    v^\beta.e_\gamma
  \end{align*}

  The matrix of the curvature tensor is then
  \begin{equation}\label{eq:3.2.1}
    \Rr^\alpha_\beta =
    \di\omega^\alpha_\beta +
    \sum_\gamma\omega^\alpha_\gamma\wedge\omega^\gamma_\beta
  \end{equation}
  which is also written as
  \begin{equation}
    \Rr = \di\omega + \omega\wedge\omega.
  \end{equation}
  The formula \cref{eq:3.2.1} provided,
  in a system of local coordinates $(x^i)$,
  \begin{equation}
    \begin{cases}
      \Rr^\alpha_{\beta,i,j} &=
      (\partial_i\Gamma^\alpha_{\beta,j}-\partial_j\Gamma^\alpha_{\beta,i}) +
      (\Gamma^\alpha_{\gamma,i}\Gamma^\gamma_{\beta,j}-
      \Gamma^\alpha_{\gamma,j}\Gamma^\gamma_{\beta,i}) \\
      \Rr^\alpha_\beta &= \sum_{i<j}\Rr^\alpha_{\beta,i,j}\di x^i\wedge\di x^j.
    \end{cases}
  \end{equation}

  The condition $\Rr^{\alpha}_{\beta,i,j}=0$ is the condition of integrability
  of the system \cref{eq:3.1.5} in the classical sense;
  it can be obtained by eliminating $v^\alpha$ from equations obtained by
  substituting \cref{eq:3.1.5} in the identity
  $\partial_i\partial_j v^\alpha = \partial_j\partial_i v^\alpha$.
\end{para}











\clearpage
\section{Differential equations of $n$-th order}
\begin{para}
  The resolution of a homogeneous linear differential equation of $n$-th order
  \begin{equation}\label{eq:4.1.1}
    \frac{\di^n}{\di x^n}y = \sum_{i=1}^n a_i(x)\frac{\di^{n-i}}{\di x^{n-i}}y.
  \end{equation}
  is equivalent to that of the system of equations of first order
  \begin{equation}\label{eq:4.1.2}
    \begin{cases}
      \frac{\di}{\di x}y_i = y_{i+1} \quad(1\le i< n),\\
      \frac{\di}{\di x}y_n = \sum_{i=1}^na_i(x)y_{n+1-i}.
    \end{cases}
  \end{equation}

  According to \S 3, this system can be described as the differential equation
  of the horizontal sections of a vector bundle of rank $n$ equipped with a
  suitable connection, and that's what we propose to explain.
\end{para}
\begin{para}\label{hypothesis:4.2}
  Let $X$ be a nonsingular complex analytic manifold purely of dimension one,
  $X_n$ the $n$-th infinitesimal neighborhood of the diagonal of $X\times X$
  and $p_1$, $p_2$ the two projections from $X_n$ to $X$.
  We denote by $\Pi_{l,k}$ the injection from $X_l$ to $X_k$, for $l\le k$.

  Let $\Omega_X^{\otimes n}$ be the $n$-th tensorial power of the invertible
  sheaf $\Omega_X^1$ ($n\in\ZZ$), hence it is also the $n$-th symmetric power.
  If $I$ is the ideal defining the diagonal of $X\times X$, we canonically have
  $I/I^2\cong\Omega_X^1$, and
  \begin{equation}
    I^n/I^{n+1} \cong \Omega_X^{\otimes n}.
  \end{equation}

  Let $\Ll$ be an invertible sheaf on $X$, we denote by $P^n(\Ll)$ the vector
  bundle of jets of sections of $n$-th order on $\Ll$.
  \begin{equation}\label{eq:4.2.2}
    P^n(\Ll) = p_{1\ast}p_2^\ast\Ll.
  \end{equation}
  The $I$-adic filtration of $P_2^{\ast}\Ll$ defines a filtration of $P^n(\Ll)$
  \begin{equation}\label{eq:4.2.3}
    \Gr^iP^n(\Ll) \cong \Gr^iP^n(\Oo)\otimes\Ll
    \cong \Omega_X^{\otimes i}\otimes\Ll\quad(0\le i\le n).
  \end{equation}

  Recall that we define by induction on $n$ the \itblue{differential operator
  of order $\le n$}: $A\colon\Mm\to\Nn$
  as being a moprhism of abelian sheaves satisfying
  \begin{itemize}
    \item if $n=0$: $A$ is $\Oo$-linear;
    \item if $n=m+1$: for any local section $f$ of $\Oo$, $[A,f]$ is of order
    $\le m$.
  \end{itemize}

  For each local section $s$ of $\Ll$, $p_2^{\ast}s$ defines a local section
  $D^n(s)$ of $P^n(\Ll)$ \cref{eq:4.2.2}.
  The $\CC$-linear morphism of sheaves $D^n\colon\Ll\to P^n(\Ll)$ is the
  universal differential operator of order $\le n$ from $\Ll$.
\end{para}

\begin{definition}
  \begin{paras}
    \item A \itblue{homogeneous linear differential equation of $n$-th order}
    on $\Ll$ is an $\Oo_X$-linear homomorphism
    $E\colon P^n(\Ll)\to\Omega_X^{\otimes n}\otimes\Ll$
    which induces the identity on the submodule
    $\Omega_X^{\otimes n}\otimes\Ll$ of $P^n(\Ll)$.
    \item A local section $s$  of $\Ll$ is a \itblue{solution} of the
    differential equation $E$ if $E(D^n(s))=0$.
  \end{paras}
\end{definition}

In fact, I cheated in this definition, in that I only consider the equations
that are put in the form ``resolute'' \cref{eq:4.1.1}.

\begin{para}\label{hypothesis:4.4}
  Suppose that $\Ll=\Oo$ and that $x$ is a local coordinate on $X$. The choice
  of $x$ permits to identify $P^k(\Oo)$ with $\Oo^{[0,k]}$, the arrow $D^k$
  becoming
  \[
  D^k\colon\Oo\To P^k(\Oo)\cong\Oo^{[0,k]}\colon
  f\mapsto(\partial_x^if)_{0\le i\le k}.
  \]
  The choice of $x$ also permits to identify $\Omega^1$ and $\Oo$, so that a
  differential equation of order $n$ is identified with a morphism
  $E\in\Hom(\Oo^{[0,k]},\Oo)$, and such has coordinates $(b_i)_{0\le i\le n}$
  with $b_n=1$. The solution of $E$ is then the (holomorphic) function $f$
  satisfying
  \begin{equation}\label{eq:4.4.1}
    \sum_{i=0}^n b_i(x)\partial_x^if=0\quad(b_n=1).
  \end{equation}
\end{para}

The theorem of existence and uniqueness of solutions to the Cauchy problem
for \cref{eq:4.4.1} indicates that
\begin{theorem}[Cauchy]
  Let $X$ and $\Ll$ be as in \ref{hypothesis:4.2}, and $E$ a differential
  equation of $n$-th order on $\Ll$. Then
  \begin{proplist}
    \item The subsheaf of $\Ll$ of solutions of $E$ is a local system $\Ll^E$
    of rank $n$ on $X$.
    \item The canonical arrow $D^{n-1}\colon\Ll^E\to P^{n-1}(\Ll)$ induces an
    isomorphism
    \[
    \Oo\otimes_\CC\Ll^E\Isom P^{n-1}(\Ll).
    \]
  \end{proplist}
\end{theorem}

It follows in particular from (ii) and \ref{thm:canonical_connection} that $E$
defines a canonical connection on $P^{n-1}(\Ll)$, whose horizontal sections are
images by $D^{n-1}$ of solutions of $E$.

\begin{para}\label{hypothesis:4.6}
  Having a differential equation $E$ on $\Ll$, we thus associated
  \begin{enumerate}[a)]
    \item a holomorphic vector bundle $\Vv$ with connection (automatically
    integrable): the bundle $P^{n-1}(\Ll)$,
    \item a surjective homomorphism \cref{eq:4.2.3} ($i=0$)
    $\lambda\colon\Vv\to\Ll$.
  \end{enumerate}

  Moreover, the solutions of $E$ are the images by $\lambda$ of horizontal
  sections of $\Vv$.
  This is just another way of expressing the passage of \ref{eq:4.1.1} and
  \ref{eq:4.1.2}.
\end{para}

\begin{para}
  Let $\Vv$ be a vector bundle of rank $n$ on $X$ equipped with a connection of
  covariant derivative $\nabla$. Let $v$ be a local section of $\Vv$ and $w$ a
  vector field on $X$, which does not vanish at any point.
  We say that $v$ is \itblue{cyclic} if the local sections $(\nabla_w)^i(v)$ of
  $\Vv$ ($0\le i< n$) form a basis of $\Vv$.
  This condition doesn't depend on the choice of $w$,
  and if $f$ is an invertible holomorphic function, then
  $v$ is cyclic if and only if $fv$ is cyclic.
  It is indeed verified by induction on $i$ that $(\nabla_{gw})^i(fv)$ lies in
  the submodule of $\Vv$ generated by $(\nabla_{w})^j(v)$ ($0\le j\le i$).

  If $\Ll$ is an invertible sheaf, we say that a section $v$ of $\Vv\otimes\Ll$
  is \itblue{cyclic} if, for every local isomorphism between $\Ll$ and $\Oo$,
  the corresponding section of $\Vv$ is cyclic.
  This applies in particular to a section $v$ of
  $\CHom(\Vv,\Ll)=\Vv^{\vee}\otimes\Ll$.
\end{para}

\begin{lemma}\label{lem:4.8}
  With hypothesis and notations in \ref{hypothesis:4.6},
  $\lambda$ is a cyclic section of $\CHom(\Vv,\Ll)$.
\end{lemma}

The problem is local on $X$; we come back to the case where $\Ll=\Oo$ and where
there exists a local coordinate $x$.

Using the notations from \ref{hypothesis:4.4}, a section $(f^i)$ of
$P^{n-1}(\Oo)\cong\Oo^{[0,n-1]}$ is horizontal if and only if it satisfies
\[
\begin{cases}
  \partial_xf^i = f^{i+1} \qquad 0\le i\le n-2 \\
  \partial_xf^{n-1} = -\sum_{i=0}^{n-1}b_if^i.
\end{cases}
\]
This gives us the coefficients of the connection: the matrix of connection is
\begin{equation}
  \begin{pmatrix}
    0 & -1 & & & \\
    & 0 & -1 & & \\
    &&\ddots&\ddots& \\
    & & & 0 & -1 \\
    b^0 & b^1 &\cdots &\cdots & b^{n-1}
  \end{pmatrix}
\end{equation}

In the chosen coordinate system, $\lambda=e^0$ and we calculate that
\[
\nabla_x^i\lambda = e^i\quad(0\le i\le n-1)
\]
what proves \ref{lem:4.8}.

\begin{proposition}\label{prop:4.9}
  The construction in \ref{hypothesis:4.6} establishes an equivalence of the
  following categories, when we take for morphisms the isomorphisms:
  \begin{enumerate}[a)]
    \item the category of invertible sheaves on $X$, equipped with differential
    equations of order $n$;
    \item the category of triples consisting of a vector bundle $\Vv$ of rank
    $n$ equipped with a connection, of an invertible sheaf $\Ll$ and
    of a cyclic homomorphism $\lambda\colon\Vv\to\Ll$.
  \end{enumerate}
\end{proposition}

Construct a quasi-inverse of the functor \ref{hypothesis:4.6}.
Let $\Vv$ be a vector bundle with connection, and $\lambda$ a homomorphism from
$\Vv$ to an invertible sheaf $\Ll$.
We denote by $V$ the local system of horizontal sections of $\Vv$. For every
$\Oo$-module $\Mm$, we have (\ref{thm:canonical_connection})
\[
\Hom_{\Oo}(\Vv,\Mm)\Isom\Hom_{\CC}(V,\Mm).
\]
In particular, we define a map $\gamma^k$ from $\Vv$ to $P^k(\Ll)$ by putting,
for every horizontal section $v$ of $\Vv$,
\begin{equation}\label{eq:4.9.1}
  \gamma^k(v) = D^k(\lambda(v)).
\end{equation}

\begin{sublemma}\label{lem:4.9.2}
  The homomorphism $\lambda$ is cyclic if and only if
  \[
  \gamma^{n-1}\colon\Vv\To P^{n-1}(\Ll)
  \]
  is an isomorphism.
\end{sublemma}

The problem is local on $X$. We come back to the case where $\Ll=\Oo$ and where
we already have a local coordinate $x$.
With the notations of \ref{hypothesis:4.4}, the morphism $\gamma^k$ admits then
for coordinates the morphisms $\partial_x^i\lambda = \nabla_x^i\lambda$
($0\le i\le k$).
For $k=n-1$, these form a basis of $\Hom(\Vv,\Oo)$ if and only if $\gamma^{n-1}$
is an isomorphism.


For $k\ge l$, the diagram
\begin{equation}\label{diag:4.9.3}
  \begin{tikzcd}
    & \Vv\ar[dl,"{\gamma^k}"']\ar[dr,"{\gamma^l}"] & \\
    P^k(\Ll)\ar[rr,"{\Pi_{l,k}}"] && P^l(\Ll)
  \end{tikzcd}
\end{equation}
is commutative; if $lambda$ is cyclic, we deduce from this fact and
\ref{lem:4.9.2} that $\gamma^n(\Vv)$ is locally a direct factor of codimension
one on $P^n(\Ll)$, and admits for supplement
$\omega^{\otimes n}\otimes\Ll\cong\ker{\Pi_{n-1,n}}$.
Then there exists one and only one differential equation of order $n$ on $Ll$:
\[
E\colon P^n(\Ll) \To \Omega^{\otimes n}\otimes\Ll
\]
such that $E\circ\gamma^n = 0$.

According to \cref{eq:4.9.1}, if $v$ is a horizontal section of $\Vv$, then
$ED^n\lambda v=E\gamma^n v=0$, so that $\lambda v$ is a solution of $E$.
Endow $P^{n-1}(\Ll)$ the connection in \ref{hypothesis:4.6} defined by $E$.
If $v$ is a horizontal section of $\Vv$, then $\gamma^{n-1}v=D^{n-1}\lambda v$,
with $\lambda v$ a solution of $E$, and thus $\gamma^{n-1}v$ is horizontal.
We deduce that $\gamma^{n-1}$ is compatible with connections.
A special case of \ref{diag:4.9.3} shows that the diagram
\[
\begin{tikzcd}
  \Vv\ar[r,"{\gamma^{n-1}}"]\ar[d,"{\lambda}"'] &
  P^{n-1}(\Ll)\ar[d,"{\Pi_{0,n-1}}"] \\
  \Ll\ar[r,equal] & \Ll
\end{tikzcd}
\]
is commutative, from where an isomorphism between $(\Vv,\Ll,\lambda)$ and the
triple deduced by \ref{hypothesis:4.6} from $(\Ll,E)$.
The functor
\[
(\Vv,\Ll,\lambda)\longmapsto(\Ll,E)
\]
is then the quasi-inverse of the functor in \ref{hypothesis:4.6}.

\begin{para}
  Summarizations of the relation between the two systems
  $(\Vv,\Ll,\lambda)$ and $(\Ll,E)$ which corresponded by \ref{hypothesis:4.6}
  and \ref{prop:4.9}.

  We have homomorphisms $\gamma^k\colon\Vv\to P^k(\Ll)$, such that
  \begin{enumerate}[({4.10.}1)]
    \item For horizontal $v$, $\gamma^k(v)=D^k\lambda(v)$.
    \item We have $\gamma^0=\lambda$ and $\Pi_{l,k}\circ\gamma^k=\gamma^l$.
    \item $\gamma^{n-1}$ is an isomorphism ($\lambda$ is cyclic).
    \item $E\gamma^n = 0$.
    \item $\lambda$ induces an isomorphism between the local system $V$ of
    horizontal sections of $\Vv$ and the local system $\Ll^E$ of solutions of
    $E$.
  \end{enumerate}
\end{para}











\clearpage
\section{Differential equations of second order}
In this section, we specialize the results of \S 4 to the case $n=2$,
and express, in a more geometric form, some of the results outlined in
R.C.Gunning's \cite{Gunning}.

\begin{para}\label{hypothesis:5.1}
  Let $S$ be an analytic space, and let $q\colon X_2\to S$ be an analytic space
  over $S$, locally isomorphic to the finite analytic space over $S$ described
  by the $\Oo_S$-algebra $\Oo_S[T]/(T^3)$.
\end{para}

  The fact that the group $\PGL_2$ acts strict-transitively three times on
  $\PP^1$ have the following infinitesimal analogy.

\begin{lemma}\label{lem:5.2}
  Under the hypothesis of \ref{hypothesis:5.1}, let $u$ and $v$ be two
  $S$-immersions from $X_2$ to $\PP_S^1$:
  \[
  \begin{tikzcd}
    X_2\ar[r,"{u}",shift left]\ar[r,"{v}"',shift right] & \PP_S^1
  \end{tikzcd}
  \]
  There exists one and only one projective (= $S$-automorphism) of $\PP_S^1$
  which transforms $u$ and $v$.
\end{lemma}

The problem is local n $S$, which permits to suppose $X_2$ is defined by the
$\Oo_S$-algebra $\Oo_S[T]/(T^3)$, and that $u(X_2)$ and $v(X_2)$ are contained
in the same affine line, saying $\AA_S^1$.
By translation, we may assume that $u(0)=v(0)=0$.
Giving $u$ amounts to choosing $f(T)\in\Oo_S[T]/(T^3)$.
The condition $u(0)=0$ implies that $f$ has zero constant term,
so the choice really boils down to $f'(0)$ and $f''(0)$.
Similarly for $v$, choosing $g(T)$.
Now one such choice is clearly, $h(T)=T(\mod T^3)$.
Let $w\colon X_2\to\AA_S^1$ denote the corresponding map.
Once we find the unique projective automorphisms mapping $w$ to $u$ and
$w$ to $v$, the unique projective automorphism mapping $u$ to $v$ follows
by composition.
Therefore, we reduce to the case $u$ is given by $f(T)=T(\mod T^3)$.
Then such a projective $p(x)$ should satisfy that
$v=p\circ u$, i.e. $f(p^\ast(T))=g(T)$.
Suppose $g(T)=eT+fT^2(\mod T^3)$, the condition is
\[
p(x)\equiv ax + bx^2\qquad (\mod{x^3}).
\]
We must then verify the existence and uniqueness of such a projective.
Since $p(0)=0$, and $p$ is written in a unique way in the form
\begin{align*}
  p(x) &= \frac{cx}{1-dx}\qquad(c\neq 0) \\
  &= cx + cdx^2\qquad (\mod{x^3}).
\end{align*}
The assertion follows immediately.

\begin{para}
  According to \ref{lem:5.2}, there exists, up to a unique isomorphism, one and
  only one couple $(u,P)$ consisting of a projective line $P$ on $S$ (with
  group structure $\PGL_2(\Oo_S)$) and an $S$-immersion $u$ from $X_2$ to $P$.
  We call $P$ the \itblue{osculating projective line} at $X_2$.
\end{para}

Let $X$ be a smooth curve, $X_2$ the second infinitesimal neighborhood of the
diagonal of $X\times X$ and $q_1$, $q_2$ the two projections from $X_2$ to $X$.

The morphism $q_1\colon X_2\to X$ is of the type considered in
\ref{hypothesis:5.1}.

\begin{definition}
  We call the projective line bundle on $X$ osculating to $q_1\colon X_2\to X$
  the \itblue{projective line bundle osculating to $X$} and denote by $P_{tg}$.
\end{definition}

By definition, we already have a canonical commutative diagram
\begin{equation}
  \begin{tikzcd}
    X_2\ar[r,hook]\ar[dr,"{q_1}"'] & P_{tg}\ar[d] \\
    & X.
  \end{tikzcd}
\end{equation}
In particular $P_{tg}$ is equipped with a canonical section $e$, image of the
diagonal section of $X_2$, and we have
\begin{equation}\label{eq:5.4.2}
  e^{-1}\Omega_{P_{tg}/X}^1 \cong \Omega_X^1.
\end{equation}

\begin{para}\label{construction:5.5}
  If $X$ is a projective line, then $\pr_1\colon X\times X\to X$ is a projective
  bundle on $X$, so that $P_{tg}$ identifies with the constant projective bundle
  of fiber $X$ on $X$, provided with the inclusion from $X_2$ to $X\times X$
  \[
  \begin{tikzcd}
    X_2\ar[r,hook]\ar[dr,"{q_1}"'] & X\times X \ar[d,"{\pr_1}"] \\
    & X.
  \end{tikzcd}
  \]

  In this particular case, we have a canonical commutative diagram
  \[
  \begin{tikzcd}
    X_3\ar[dr,hook]\ar[ddr] & \\
    X_2\ar[u,hook]\ar[r,hook]\ar[dr,"{q_1}"'] & P_{tg} \ar[d] \\
    & X.
  \end{tikzcd}
  \]
\end{para}

Let $X$ be a smooth curve again.
\begin{definition}[local form]\label{defn:5.6}
  A \itblue{projective connection} on $X$ is a sheaf on $X$ of germs of local
  isomorphisms from $X$ to $\PP^1$, that is a homogeneous principal sheaf
  (=torsor) under the constant sheaf of groups with value $\PGL_2(\CC)$.
\end{definition}

If $X$ is provided a projective connection, then every local construction on
$\PP^1$, invariant under the projective group, can be transported to $X$; in
particular, the construction \ref{construction:5.5} provides us a morphism
$\gamma$ inserted in a commutative diagram
\begin{equation}\label{diag:5.6.1}
  \begin{tikzcd}
    X_3\ar[dr,hook,"{\gamma}"]\ar[ddr] & \\
    X_2\ar[u,hook]\ar[r,hook]\ar[dr,"{q_1}"'] & P_{tg} \ar[d] \\
    & X.
  \end{tikzcd}
\end{equation}
It is not difficult to verify that such a morphism $\gamma$ is defined by one
and only one projective connection (a demonstration will be given in 5.10), so
that the \cref{defn:5.6} is equivalent to the following.
\setcounter{definition}{5}
\begin{definition}[infinitesimal form]
  A \itblue{projective connection} on $X$ is a morphism
  $\gamma\colon X_3\to P_{tg}$ making the diagram \cref{diag:5.6.1} commutative.
\end{definition}

Intuitively, giving a projective connection (infinitesimal form) permits to
define the birapport (= anharmonic ratio) of $4$ infinitesimally near points
one thus has to define the blrapport of $4$ neighboring points (local form).

\begin{para}
  Put $\Omega^{\otimes n}=(\Omega_X^1)^{\otimes n}$.
  The ideal sheaf on $X_3$ defining $X_2$ is canonically isomorphic to
  $\Omega^3$ and is killed by the ideal sheaf defining the diagonal.
  According to this, if $\Delta$ is the diagonal map, we have \cref{eq:5.4.2}
  \[
  \Delta^{-1}\gamma^{-1}\Omega_{P_{tg}/X}^1\cong\Omega^1.
  \]
\end{para}









\clearpage
\section{Multivalued functions of finite determination}
\begin{para}\label{hypothesis:6.1}
  Let $X$ be a non-empty connected, locally path-connected and locally simply
  connected topological space, and let $x_0$ be a point of $X$.
  We denoted by $\widetilde{X}_{x_0}$ the universal covering of $(X,x_0)$ and by
  $\widetilde{x_0}$ the base point of $\widetilde{X}_{x_0}$.
\end{para}

If $\Ff$ is a sheaf on $X$, we put
\begin{definition}
  A \itblue{multivalued section} of $\Ff$ on $X$ is a global section of the
  pullback $\pi^{-1}\Ff$ of $\Ff$ on $\widetilde{X}_{x_0}$.
\end{definition}

If $s$ is a multivalued section of $\Ff$ on $X$, a \itblue{determination} of
$s$ at a point $x$ of $X$ is an element of the stalk $\Ff_{(x)}$ of $\Ff$ at
$x$ defined by pullback of $s$ by a local section of $\pi$ at $x$.
Each point of $\pi^{-1}(x)$ defines such a determination of $s$ at $x$.
The \itblue{determination of base} of $s$ at $x_0$ is the determination defined
by $\widetilde{x_0}$.
The \itblue{determination} of $s$ on an open $U$ of $X$ is a section of $\Ff$
on $U$ whose germ at each point is the determination of $s$ at that point.

\begin{definition}
  We say that $\Ff$ satisfies the \itblue{principle of analytic continuation}
  if the coincidence place of two local sections of $\Ff$ is always (open and)
  closed.
\end{definition}
\begin{example}
  If $\Ff$ is a coherent analytic sheaf on a complex analytic space,
  $\Ff$ satisfies the principle of analytic continuation if and only if
  $\Ff$ is without immerged components.

  Let $X$ be the open unit disk $\{|z|<1\}$ and $\Ff=\Oo_X/(z)$.
  Then the sections of $\Ff$, $u=0$ and $v=$ anything not vanishing at $0$
  violate the principle of analytic continuation since the locus where they
  coincide is the punctured disc, not closed in $X$.
\end{example}


\begin{proposition}\label{prop:6.5}
  Let $X$ and $x_0$ as in \ref{hypothesis:6.1} and $\Ff$ a sheaf of $\CC$-vector
  spaces on $X$ satisfying the principal of analytic continuation. For every
  multivalued section $s$ of $\Ff$, the following conditions are equivalent.
  \begin{proplist}
    \item The determinations of $s$ at $x_0$ generate a finite dimensional
    subspace of $\Ff_{x_0}$.
    \item The subsheaf of $\CC$-vector spaces of $\Ff$ generated by the
    determinations of $s$ is a complex local system.
  \end{proplist}
\end{proposition}

(ii) $\then$ (i) is trivial.
Let's prove (i) $\then$ (ii).
Let $x$ be a point of $X$ at which the determinations of $s$ generate a finite
dimensional subspace of $\Ff_x$ and let $U$ be a connected open neighborhood of
$x$ above which $\widetilde{X}_{x_0}$ is trivial:
$(\pi^{-1}(U),\pi)\cong(U\times I,\pr_1)$ for some suitable set $I$.
This implies on $U$ that the determinations of $s$ generate a complex local
system: each $i\in I$ defines a determination $s_i$ of $s$ and on $U$ the
subsheaf of vector spaces of $\Ff$ generated by the determinations of $s$ is
generated by $(s_i)_{i\in I}$; if this sheaf is constant, the hypothesis on $x$
implies that it is a complex local system. We have
\begin{lemma}
  If a sheaf of $\CC$-vector spaces on $\Ff$ on a connected space satisfies the
  principal of analytic continuation, then the subsheaf of vector spaces of
  $\Ff$ generated by a family of global sections $s_i$ is a constant sheaf.
\end{lemma}

The sections $s_i$ define
\[
a\colon\underline{\CC}^{(I)}\To\Ff
\]
with image the subsheaf of vector spaces $\Gg$ of $\Ff$ generated by $s_i$.
If a relation $\sum\lambda_is_i=0$ is satisfied at a point, it is true
everywhere by the principal of analytic continuation.

The sheaf $\ker(a)$ is then a constant subsheaf of $\underline{\CC}^{(I)}$ and
the assertion follows.

We conclude the demonstration of \ref{prop:6.5} by note that, according to the
above, the largest open of $X$ on which the determinations of $s$ generate a
local system is closed and contains $x_0$.

\begin{definition}
  Under the hypothesis of \ref{prop:6.5}, a multivalued section $s$ of $\Ff$ is
  \itblue{of finite determination} if it satisfies the equivalent conditions in
  \ref{prop:6.5}.
\end{definition}

\begin{para}\label{hypothesis:6.8}
  Under the hypothesis of \ref{prop:6.5}, let $s$ be a multivalued section of
  finite determination of $\Ff$. This section defines
  \begin{enumerate}[a)]
    \item the local system $V$ generated by its determinations;
    \item a germ of sections of $V$ at $x_0$, saying $v_0$, corresponding to
    the determination of base of $s$;
    \item an inclusion $\lambda\colon V\to\Ff$.
  \end{enumerate}

  The triple form of $V_{x_0}$, of $v_0$ and of the representation of
  $\pi_1(X,x_0)$ on $V_{x_0}$ defined by $V$ (\ref{cor:1.4})
  is called the \itblue{monodromy} of $s$.
  The triple $(V,v_0,\lambda)$ satisfies the following two conditions.
  \begin{subpara}
    $v_0$ is a cyclic vector of the $\pi_1(X,x_0)$-module $V_{x_0}$, i.e.
    generates the $\pi_1(X,x_0)$-module $V_{x_0}$.
  \end{subpara}

  This simply means that $V$ is generated by the set of determinations of the
  unique multivalued section of $V$ having determination of base $v_0$.

  \begin{subpara}
    \[
    \lambda\colon V_{x_0}\To\Ff_{x_0}
    \]
    is injective.
  \end{subpara}
\end{para}

\begin{para}
  Let $W_0$ be a finite dimensional complex representation of $\pi_1(X,x_0)$
  equipped with a cyclic vector $w_0$. The multivalued section $s$ of $\Ff$ is
  said of \itblue{subordinate monodromy} at $(W_0,w_0)$ if it is of finite
  determination and if, with the notations in \ref{hypothesis:6.8}, there exists
  a homomorphism of $\pi_1(X,x_0)$-representations from $W_0$ to $V_{x_0}$
  sending $w_0$ to $v_0$. Let $W$ be the local system defined by $W_0$, and $w$
  the unique multivalued section of $W$ of determination of base $w_0$.
  It is clear that, under the hypothesis of \ref{prop:6.5}, we have
\end{para}
\begin{proposition}
  The function $\lambda\mapsto\lambda(w)$ is a bijection between
  $\Hom_{\CC}(W,\Ff)$ and the set of multivalued sections of $\Ff$ of
  subordinate monodromy at $(W_0,w_0)$.
\end{proposition}
\begin{corollary}
  Let $X$ be a connected reduced complex analytic space
  equipped with a base point $x_0$,
  $W_0$ a finite dimensional complex representation of $\pi_1(X,x_0)$
  equipped with a cyclic vector $w_0$,
  $W$ the local system defined by $W_0$,
  $\Ww=\Oo\otimes_{\CC}W$ the associated vector bundle,
  $w$ the unique multivalued section of $\Ww$ of determination of base $w_0$,
  and $\Ww^{\vee}$ the dual vector bundle of $\Ww$.
  The function
  \[
  \lambda\longmapsto\<\lambda,w\>
  \]
  from $\Gamma(X,\Ww^{\vee})$ to the set of multivalued holomorphic functions
  on $X$ of subordinate monodromy at $(W_0,w_0)$, is a bijection.
\end{corollary}

\begin{corollary}
  If $X$ is Stein, there exist multivalued holomorphic functions on $X$
  having any monodromy $(W_0,w_0)$ given in advance.
\end{corollary}

If $X$ is Stein, then any coherent sheaf on it is generated by global sections.
Then the bijection gives the desired data.

\begin{bibdiv}
\begin{biblist}
\bibselect{refdatabase}
\end{biblist}
\end{bibdiv}
\end{document}
